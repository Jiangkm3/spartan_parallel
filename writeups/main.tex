\documentclass{article}
\usepackage[letterpaper, total={6in, 8in}]{geometry}
\usepackage{xcolor}
\usepackage{amsfonts}

\newcommand{\red}[1] {\color{red}#1\color{black}}
\newcommand{\code}{\texttt}
\newcommand{\Qsum}{Q_{\mathtt{sum}}}
\newcommand{\Qmax}{Q_{\mathtt{max}}}
\newcommand{\qmax}{q_{\mathtt{max}}}
\renewcommand{\P}{\mathcal{P}}
\newcommand{\V}{\mathcal{V}}
\newcommand{\F}{\mathbb{F}}
\newcommand{\eq}{\widetilde{\mbox{eq}}}
\newcommand{\valid}{\tilde{\mbox{valid}}}

\title{Spartan Parallel}
\author{Kunming Jiang}
\date{Sep 21, 2023}

\begin{document}

\maketitle

\section{Introduction}\label{intro}

The problem is consisted of $P$ R1CS instances of the form $\{A_i, B_i, C_i\}$, where each $A_i, B_i, C_i$ are $X \times Y$ matrices, $X$ represents the number of constraints and $Y$ represents the number of (inputs + witnesses). For each instance $\{A_i, B_i, C_i\}$, there are $Q_i$ (input, witnesses) vectors of length $Y$ that claim to be a correct execution of the instance. We name these (input, witnesses) vectors $z_{i, 0},\dots z_{i, Q_i}$. The goal is to prove the claim for all such $z_{i, j}$ using SNARK.

For simplicity, assume all $P, Q_i, X, Y$ are powers of two. Define $\Qmax \leftarrow \max_i Q_i$ and $\Qsum \leftarrow \sum_i Q_i$. Let $p = \log P$, $q_i = \log Q_i$, $\qmax = \log \Qmax$, $x = \log X$, and $y = \log Y$. We want the total runtime of the Prover ($\P$) to be $O(\Qsum \cdot X)$, and that of the Verifier ($\V$) to be $O(\log(P \cdot \Qmax \cdot Y))$.

\section{Starting from Spartan}\label{spartan}
We begin by modifying Spartan to support data-parallelism. In section \ref{identify} we perform a complexity analysis and in \ref{reduce-time} we resolve the complexity blow-up.

The modified protocol is shown below:
\begin{enumerate}
    \item $\P$ commits witnesses and sends the commit $\mathcal{C}$ to $\V$.
        \label{step:witness-commit}
    \item $\V$ samples $\tau\in_R\F^{p + \qmax + x}$ and sends $\tau$ to $\P$.
        \label{step:sample-tau}
    \item $\P$ computes $Az, Bz, Cz, \eq_\tau: \F^{p + \qmax + x} \to \F$, defined as:
        $$Az(t_p, t_q, t_x) = A(t_p, t_x) \cdot z(t_p, t_q, t_x)$$
        $$\eq_\tau(t) = 1\mbox{ if }t = \tau,\ 0\mbox{ otherwise }$$
        \label{step:mat-product}
    \item $\V$ and $\P$ perform sum-check \#1 to prove
        $$\displaystyle\sum_{t\in \F^{p + \qmax + x}}(Az(t)\cdot Bz(t) - Cz(t))\cdot \eq_\tau(t) = 0$$
        \label{step:sumcheck-1}
    \item By the end of the sum-check, $\V$ samples $r = (r_p, r_q, r_x)\in_R\F^{p + \qmax + x}$ and $\P$ obtains $v_A = Az(r)$, $v_B = Bz(r)$, $v_C = Cz(r)$. $\P$ sends $v_A, v_B, v_C$ to $\V$.
    \item $\V$ checks $(v_A\cdot v_B - v_C)\cdot \eq_\tau(r) = e_1$, where $e_1$ is the final claim of the sum-check.
        \label{step:sumcheck-1-verify}
    \item $\V$ samples $r_A, r_B, r_C\in_R\F$ and sends to $\P$. $\P$ computes $T = r_A\cdot v_A + r_B\cdot v_B + r_c\cdot v_C$.
    \item $\P$ computes 
        $$ABC: \F^{p + y} \to \F = r_A \cdot A(\cdot) + r_B \cdot B(\cdot) + r_C \cdot C(\cdot)$$
        $$Z_{r_q}: \F^{p + y}\to\F \mbox{ where } Z_{r_q}(t_p, t_x) = z(t_p, r_q, t_x)$$
        $$\eq_{r_p}(t_p) = 1 \mbox{ if } t_p = r_p\mbox{, 0 otherwise }$$
        \label{step:compute-z-poly}
    \item $\V$ and $\P$ perform sum-check \#2 to prove
        $$\displaystyle\sum_{t\in \F^{p + y}}ABC(t)\cdot Z_{r_q}(t)\cdot \eq_{r_p}(t_p) = T$$
        By the end of the sum-check, $\V$ samples $r^* = (r_p^*, r_x)\in_R\F^{p + y}$ and obtains the claim $e_2$.
        \label{step:sumcheck-2}
    \item $\P$ opens the commitment $\mathcal{C}$ and evaluates the witness polynomial on $r_p^*, r_q, r_y[1..]$ as $w$, sends $w$ to $\V$.
        \label{step:evaluate-witness}
    \item $\V$ evaluates the input polynomial on $r_p^*, r_q, r_y[1..]$ as $v$, computes $v_Z = (1 - r_y[0])\cdot w + r_y[0]\cdot v$. $\V$ also evaluates $v_{r_p} = \eq_{r_p}(r_p^*)$
        \label{step:evaluate-input}
    \item $\V$ checks $e_2 = (r_A\cdot v_A + r_B\cdot v_B + r_c\cdot v_C)\cdot v_Z \cdot v_{r_p}$.
\end{enumerate}

\section{Identifying Runtime Overhead}\label{identify}
\subsection{Verifier Cost}
Since we only require the Verifier cost to be $O(\log(P \cdot \Qmax \cdot Y))$. This has already been achieved by the above protocol, so there isn't any improvement we need to do. Here's the cost breakdown:
\begin{itemize}
    \item In step \ref{step:sample-tau}: $O(\log(P \cdot \Qmax \cdot Y))$ for sampling $\tau$.
    \item In step \ref{step:sumcheck-1}: $O(\log(P \cdot \Qmax \cdot Y))$ due to the sum-check having $p + \qmax + x$ rounds. Assume $X$ and $Y$ are of similar size.
    \item In step \ref{step:sumcheck-1-verify}: $O(\log(P \cdot \Qmax \cdot Y))$ for evaluating $\eq_\tau(r)$.
    \item In step \ref{step:sumcheck-2}: $O(\log(P \cdot Y))$ due to the sum-check having $p + y$ rounds.
    \item In step \ref{step:evaluate-input}: $O(\log(P \cdot \Qmax \cdot Y))$ for evaluating the input, and $O(\log(P))$ for evaluating $\eq_{r_p}(r_p^*)$.
\end{itemize}

\subsection{Prover Cost}
The prover's cost is currently $O(P\cdot\Qmax\cdot Y\cdot)$, and we need to reduce it to $O(\Qsum \cdot X)$. Here's all the steps that exceeds that complexity:
\begin{itemize}
    \item Step \ref{step:witness-commit}: witness commit
    \item Step \ref{step:mat-product}: producing $Az$, $Bz$, $Cz$, and $\eq_\tau$
    \item Step \ref{step:sumcheck-1}: sum-check 1
    \item Step \ref{step:compute-z-poly}: compute $Z_{r_q}$
\end{itemize}
We now discuss how to improve Prover's cost in these steps.

\section{Reducing Time Complexity}\label{reduce-time}
Here are the general ideas on how to improve time complexity:
\begin{itemize}
    \item We assume that filling an array / matrix with 0s does not take time. This assumption helps us reduce the time complexity of the following steps:
    \begin{itemize}
        \item Step \ref{step:witness-commit}: to perform witness commit, we need to express witnesses as a $\F^{p + \qmax + x}\to\F$ dense polynomial. By the assumption, we can initialize this matrix to 0, and then fill-in the non-zero elements. This reduces the polynomial construction cost to $O(\Qsum \cdot X)$, which satisfies the time complexity.
        \item Step \ref{step:mat-product}: Using the similar idea, we can also initialize $Az$, $Bz$, $Cz$ as zero-polynomials and then fill in their non-zero entries. This reduces their cost to $O(\Qsum \cdot X)$.
    \end{itemize}
    \item Next, sum-check protocols can be easily modified to "skip" the zero parts, if these zero parts are clustered together in the polynomial. As a result, assume that the polynomials has $O(\Qsum \cdot X)$ non-zero entries and the zeros are clustered (in our case, cluster means that for certain $p$ and $q$, the polynomial always evaluates to 0 regardless of the choices of $y$), we can reduce the time complexity of sumchecks. We already show that this is the case for $Az$, $Bz$, and $Cz$, so what remaining is to reason about $\eq$.
    \item While at the first glance it seems like one can treat $\eq$ just like every other polynomial: initialize with 0 and fill in non-zero parts. We call this modified version $\eq^*$. However, such modification would result in $\V$ unable to compute the binding in logarithmic time. Instead, we do the following:
    \begin{itemize}
        \item Let $\valid:\F^{p + \qmax}\to \{0, 1\}$ be multilinear extension of the indicator function that evaluates to 1 if $(p, q)$ is a valid instance-(input, witness) index pair, 0 otherwise.
        \item The verifier, however, still cannot evaluate $\valid$ in logarithmic time. This is because since we make no assumption on $Q_i$'s and how they relate to $P$, expressing $\valid$ would take at least $O(P)$ time.
        \item Thus, $\P$ needs to commit to $\valid$ at the beginning of the protocol. $\P$ can then use $\eq^*$ during sumcheck, and $\V$ use $\eq \cdot \valid$ to bind $\eq^*$ to $r_p$ and $r_q$.
    \end{itemize}
    \item Improving the time complexity of computation on $Z_{r_q}$ is more tricky. Since $r_q$ is completely random and $Z_{r_q}$ needs to be evaluated on $P \times X$ points on the boolean hypercube, this does not give us much room to wiggle. The best solution (currently) is to evaluate $Q$ in reverse order, i.e. start from the least siginificant bit. This allows us to keep skipping invalid $p, q$ pairs since the most significant bits of $q$ are still within the boolean hypercube. The only problem is that once $q$ is reduced to 1, we can no longer cut the evaluation size by half. Overall, this reduces the time complexity of computing $Z_{r_q}$ to $O(\Qsum\cdot \log(P\cdot\Qmax) \cdot X)$.
\end{itemize}

\section{Reducing Space Complexity}\label{reduce-space}

Idea: express $Z$ as a polynomial over $p$ times a polynomial over $q$ over a polynomial over $x$

\end{document}